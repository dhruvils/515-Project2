\documentclass[12pt]{article}
\usepackage{amsmath}
\begin{document}
This system arises from the fact that in our points, we have uniform knot multiplicity. 
Technically, the system giving the control points from the points we wish to use to 
generate a cubic spline curve is given by a tridiagonal matrix found from using the 
De Boor algorithm. Since the beginning and end points are known, they are omitted from 
the left side of the equality, since the De Boor algorithm leaves two degrees of freedom in the solution.
The usage of the De Boor algorithm was to allow continuity between the joined splines up to the second derivative.

\begin{align*}
&\frac{d^2}{dt^2} (1-t)^3 b_0 + (1-t)^2 t b_1 + (1-t)t^2 b_2 + t^3 b_3\\
&= 2(-3 b_0(t-1) + b_1(3t - 2) - 3 b_2 t + 3 b_3 t + b_2)\\
&= -6 b_0 t + 6 b_1 t - 6 b_2 t + 6 b_3 t + 6 b_0 - 4 b_1 + 2 b_2)\\
\end{align*}

Evaluated at $t=0$,
$C_i(0) = 6 b_0^i - 4 b_1^i + 2 b_2^i$.
Evaluated at $t=1$,
$C_i(1) = 2 b_1^i - 4 b_2^i + 6 b_3^i$.

We then have the constraints that 
$6 b_0^{i-1} - 4 b_1^{i-1} + 2 b_2^{i-1} = 2 b_1^i - 4 b_2^i + 6 b_3^i$ and
$6 b_0^{i} - 4 b_1^{i} + 2 b_2^{i} = 2 b_1^{i+1} - 4 b_2^{i+1} + 6 b_3^{i+1}$,
as well as $b_3^{i-1} = b_0^i$ and $b_3^i = b_0^{i+1}$.
Manipulating the for equation for all $i$, we have 
which leads to, after a little more manipulation
\begin{align*}
6b_3^i &= 6 b_0 ^{i-1} - 4 b_1^{i-1} + 2b_2^{i-1}-2b_1^{i}+4b_2^i\\
&= d_{i-2} + 4 d_{i-1} + d_{i}\\
6b_0^i &= d_{i-1} + 4 d_{i} + d_{i+1}
\end{align*}
\end{document}
